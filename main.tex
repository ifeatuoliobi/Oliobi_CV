% --- LaTeX CV Template - S. Venkatraman ---

% Set document class and font size
\documentclass[letterpaper, 11pt]{article}
\usepackage[utf8]{inputenc}

% Package imports
\usepackage{setspace, longtable, graphicx, hyphenat, hyperref, fancyhdr, ifthen, everypage, enumitem, amsmath, setspace}

% --- Page layout settings ---

% Set page margins
\usepackage[left=1in, right=1in, bottom=0.7in, top=0.7in]{geometry}

% Set line spacing
\renewcommand{\baselinestretch}{1.15}

% --- Page formatting ---

% Set link colors
\usepackage[dvipsnames]{xcolor}
\hypersetup{colorlinks=true, linkcolor=Blue, urlcolor=Blue}

% Set font to Libertine, including math support
\usepackage{libertine}
\usepackage[libertine]{newtxmath}

% Remove page numbering
\pagenumbering{gobble}

% --- Document starts here ---

\begin{document}

% Name and date of last update to this document
\noindent{\Huge{\color{Blue}{\href{https://www.ifeatuoliobi.com/}{Ifeatu Oliobi}}}
\hfill}

% --- Contact information and other items ---

\begin{center}
\begin{tabular}{l l}
Teachers College, Columbia University    &  \\
Department of Education Policy and Social Analysis    &  \hspace{1in} Email: \href{mailto:ifeatu.oliobi@gmail.com}{ifeatu.oliobi@gmail.com}  \\
525 W 120th Street             &  \hspace{1in}  Website: \href{www.ifeatuoliobi.com}{www.ifeatuoliobi.com} \\
New York, NY - 10027 & \hspace{1.4in} Phone: +1 (732) 881-6077 \\
\end{tabular}
\end{center}

\vspace{1em}

% --- Start the two-column table storing the main content ---

% Set spacing between columns
\setlength{\tabcolsep}{8pt}

% Set the width of each column
\begin{longtable}{p{1.2in}p{4.8in}}

% --- Section: Education ---

{\textbf{EDUCATION}} 
& \textbf{Teachers College, Columbia University} \hfill May 2023 (expected) \\ 
& Ph.D. Candidate in Economics and Education \hfill  \\
&  {\it Dissertation Title: ``Essays in the Economics of Education''}\\
& \\

& \textbf{University of Warwick} \hfill 2011 \\
& M.Sc. in Economics \hfill  \\
& \\

& \textbf{Covenant University} \hfill 2009 \\
& B.Sc. in Economics \hfill  \\
& \\

% --- Section: Research interests ---

\nohyphens{{\textbf{RESEARCH FIELDS}}}
& Economics of Education, Development Economics, Applied Microeconomics  \\
\\

% --- Section: Awards, scholarships, etc. ---
% --- Note: section title is spread over two lines ---

{{\textbf{SCHOLARSHIPS}}} 
& Research Grant Award, Teachers College, Columbia University \hfill 2022\\
{{\textbf{AND}}} 
& Provosts Grant, Teachers College, Columbia University \hfill 2022 \\
{{\textbf{FELLOWSHIPS}}} 
& Burke Scholarship, Teachers College, Columbia University \hfill 2018 -- 2020 \\
& Doctoral Fellowship, Teachers College, Columbia University \hfill 2017 -- 2020 \\
& Minority Scholarship, Teachers College, Columbia University \hfill 2017 -- 2020 \\
& \\

% --- Section: Research experience ---

\nohyphens{{\textbf{RESEARCH}} 
& Ithaka S+R, \textit{Researcher} \hfill 2022-- Present \\
\textbf{{EXPERIENCE}}} 
& New York City Department of Education, \textit{Research Consultant}   \hfill 2019 -- 2020 \\ 
\textbf{{AND OTHER}}
&   Teachers College, Columbia University, \textit{RA for Peter Bergman}  \hfill 2020 -- 2022 \\
\textbf{{EMPLOYMENT}}
& Teachers College, Columbia University, \textit{RA for Alex Eble} \hfill 2017 -- 2020 \\ 
& Oxford Policy Management, \textit{Education Consultant}  \hfill 2014 -- 2018 \\
& South Sudan Ministry of Finance, \textit{Economist}  \hfill 2011 -- 2014 \\
& Central Bank of Nigeria, \textit{Assistant Economist}  \hfill 2009 -- 2010 \\
& \\


% --- Section: Teaching experience ---

{\textbf{{TEACHING}} 
& Economics of Public Finance [Undergraduate] \hfill Summer 2021 \\ \textbf{{EXPERIENCE}}} 
& \textit{TA for Liz Ananat, Barnard College, Columbia University} \hfill  \\  \\
& Education and Economic Development  [Masters] 
\hfill Spring 2019 \\  
& \textit{TA for Alex Eble, Teachers College, Columbia University} \hfill \\   \\
& Microeconomic Theory Applications to Education  [Masters]  
\hfill Fall 2018 \\ 
& \textit{TA for Alex Eble, Teachers College, Columbia University} \hfill \\ \\




% --- Section: Publications ---
\nohyphens{{\textbf{RESEARCH PAPERS}}} 
& \textbf{\color{Blue}Access to Higher Education and Family Formation: Evidence from University Expansion in Nigeria } (\textit{\textbf{Job Market Paper)}} \\
& \textit{Abstract} \\
& How do greater education opportunities impact family formation? This paper evaluates a rapid expansion in public universities in Nigeria to estimate the impact of increased access to education on family formation in a low-income context. My empirical analysis combines administrative and survey data from Nigeria with a new staggered difference-in-differences estimator that exploits the geographical and time-wise variation in the university expansion. I show that greater higher education opportunities led to increased years of schooling and educational attainment among school-aged women and delayed the timing of first marriage and childbirth. In addition, university openings reduced the number of births these women had and increased the likelihood of better health outcomes for their children. I find suggestive evidence that these outcomes are driven by the effects of education on women's knowledge and autonomy - women delay the onset of sexual activity, and there is increased use of contraceptive methods, labor force participation, and intra-household bargaining power.
& \\

& \textbf{\color{Blue}Female Schooling and Marriage Outcomes: Evidence from Nigeria’s Universal Primary Education Policy} \\
& \textit{Abstract} \\
& This paper evaluates the impact of schooling expansion reforms on female education and marriage outcomes in Nigeria. Using evidence from Nigeria’s 1976 universal primary education reform, I implement a difference-in-differences design that exploits the variation in exposure to the reform across birth cohorts and localities. I find that the reform significantly increased educational attainment for women. Women with more schooling are less likely to be married, delay marriage, and are more likely to be in polygamous unions. The spousal education gap increases but the reforms do not significantly impact the likelihood that a woman experiences domestic violence. 
& \\

& \textbf{\color{Blue}Firm Culture: Examining the Role of Gender and Ethnicity in Job Matching in an Online African Labor Market}  \textit{with Belinda Archibong, Francis Annan and Anja Benshaul-Tolonen}. \\
& \textit{Abstract} \\
& Africa has some of the highest rates of unemployment globally. While a growing literature has linked ethnicity and ethnic bias to inefficient outcomes across a range of contexts in Africa, there is limited understanding of their contribution to labor market frictions and unemployment. Using new administrative data on 194,190 applicants and over 1.3 million matches from the largest online job platform in Nigeria, we study the role of ethnicity in matching and firm hiring. We find significant differences in the matching outcomes of applicants by ethnicity. While on average, applicants that share the same ethnicity as hiring managers from ethnic majority groups are more likely to be hired, the effects differ significantly by the applicant’s gender. Co-ethnic female applicants are much less likely to be hired by hiring managers, while co-ethnic men are more likely to be hired. Male and female hiring managers exhibit similar hiring behavior by ethnicity.
& \\

% --- Section: Publications ---
\nohyphens{{{\textbf{WORK IN PROGRESS}}}} 
& \textbf{\color{Blue}'Cheap Talk?’ The Effects of Information Interventions on Gender Gaps in Online Labor Markets} \href{https://www.socialscienceregistry.org/trials/8841}{RCT: AEARCTR-0008841} \textit{with Belinda Archibong, Francis Annan, and Oyebola Okunogbe}. \\ \\
& \textbf{\color{Blue}Motivating Teacher Effort in Kenyan Private Schools} \textit{with Alex Eble and Tim Sullivan}. \\
& \\



% --- Section: Talks and tutorials ---

{\textbf{PROFESSIONAL}}
& 46th Annual Association for Education Finance Policy (AEFP) \hfill 2022 \\
{\textbf{ACTIVITIES}}
& 91st Annual Meeting of the Southern Economic Association (SEA) \hfill 2021 \\
& 45th Annual Association for Education Finance Policy (AEFP) \hfill 2021 \\
& 3rd Annual UNU-WIDER Jobs and Development Conference \hfill 2020 \\ \\

% --- Section: Various skills (programming, software, languages, etc.) ---

{\textbf{SKILLS}}
& \textbf{Languages/Software}: Stata, R, LaTeX, Qualtrics, Nvivo  \\
& \textbf{Tools}: Econometrics, Causal Inference, Applied Statistics, Survey Design \\
& \\

{\textbf{PERSONAL}}
& \textbf{Citizenship}: United States, Nigeria  \\
& \\

% --- Section: References-
{\textbf{REFERENCES}}
\\ \\
 \begin{tabular}{@{}p{0in}p{3.0in}p{3.0in} l l l}
     & \href{http://www.alexeble.com}{Alex Eble} & \href{http://https://sites.google.com/tc.columbia.edu/jscottclayton}{Judith Scott-Clayton} \\
 & Dept. of Education Policy and Social Analysis  &  Dept. of Education Policy and Social Analysis   \\
 & Teachers College, Columbia University & Teachers College, Columbia University \\
 & \small{\href{mailto:eble@tc.columbia.edu}{eble@tc.columbia.edu}} & \small{\href{mailto:scott-clayton@tc.columbia.edu}{scott-clayton@tc.columbia.edu}} \\
&& \\
 & \href{http://https://sites.google.com/view/belinda-archibong}{Belinda Archibong} &  \\
 & Department of Economics & \\
 & Barnard College, Columbia University &   \\
 & \small{\href{mailto:ba2207@columbia.edu}{ba2207@columbia.edu}} &  \\
\end{tabular}




% --- End of CV! ---

\end{longtable}
\end{document}
